
\documentclass[11pt,a4paper]{article}
%\usepackage{russ}
\usepackage[utf8]{inputenc}
\usepackage[english,russian]{babel}

\usepackage{amsmath, amssymb}
\usepackage{amscd}
\usepackage{amsthm}
% \usepackage[left=1.5cm,right=1.5cm,top=1cm,bottom=1cm]{geometry}
\usepackage{geometry}
\usepackage{fancyhdr}
\usepackage{color}
\usepackage[pdftex,dvipsnames]{xcolor}  % Coloured text etc.
\usepackage{graphicx}
\usepackage{xargs}                      % Use more than one optional parameter in a new 
\graphicspath{ {./pic/} }

\def\Z{\mathbb{Z}}
\def\N{\mathbb{N}}
\def\R{\mathbb{R}}
\def\F{\mathbb{F}}
\def\Q{\mathbb{Q}}

\def\E{\mathrm{\mathbb{E}}}
\def\D{\mathrm{\mathbb{D}}}
\def\P{\mathrm{P}}

\newcounter{znum}
\newcommand{\z}[1]{\addtocounter{znum}{1} \textbf{Задача \arabic{znum}#1. }}

\newcounter{defnum}
\newcommand{\df}[1]{\addtocounter{defnum}{1} \textbf{Определение \arabic{defnum}.} {\it #1}}

\renewcommand{\!}{\textcolor{red}{!}}

\usepackage{wrapfig}
\usepackage{graphicx} % Allows including images
\usepackage{float}


\usepackage[
	% disable,
	colorinlistoftodos, prependcaption, textsize=normalsize,
	linecolor=Dandelion, backgroundcolor=Yellow!25, bordercolor=Dandelion,
	textwidth=100
]{todonotes}

\newcommandx{\note}[2][1=]{\todo[size=footnotesize, #1]{#2}}
\newcommandx{\longnote}[2][1=]{\todo[inline, nolist, linecolor=Gray, backgroundcolor=Gray!25, bordercolor=Gray, #1]{#2}}
\newcommandx{\problem}[2][1=]{\todo[backgroundcolor=red!25, linecolor=red, bordercolor=red, #1]{#2}}
\newcommandx{\question}[2][1=]{\todo[
	noprepend, inline,
	linecolor=RoyalBlue, backgroundcolor=RoyalBlue!25, bordercolor=RoyalBlue, #1]{#2}
}
\newcommandx{\thought}[2][1=]{\todo[
	noprepend, inline,
	linecolor=Orchid, backgroundcolor=Orchid!25, bordercolor=Orchid, #1]{#2}
}

\newcommandx{\checkref}[2][1={check reference validity}, 2=]{
	\todo[linecolor=red, backgroundcolor=red!25, bordercolor=red, size=footnotesize, #2]{#1}
}
\newcommandx{\needref}[2][1={need reference}, 2=]{
	\todo[linecolor=red, backgroundcolor=red!25, bordercolor=red, size=footnotesize, #2]{#1}
}
\newcommand{\EMPH}[1]{{\large \textit{\textbf{#1}}}}


\begin{document}

\pagestyle{empty}

\begin{center}
	\large
	\textbf{Затравочные задачи.}\\
\end{center}

\z{} Одновременно подбрасываются 2 монеты, какова вероятность того что выпадет хотя бы один орел?

\z{} Одновременно кидаются 2 шестигранных кубика. Какая сумма чисел выпавших на кубиках более вероятна: 9 или 10?

\z{} В одной из трех коробок лежит приз, две другие коробки пустые. Вы можете показать на одну из коробок, после чего для вас откроют другую коробку, в которой ничего нет. После этого вам предлагается окончательно выбрать коробку. Можете ли вы выиграть приз с вероятностью, большей 1/2?

\z{} В мешке есть три карты. У одной обе стороны белые, у другой --- красные, а у третьей --- одна белая и одна красная. Из мешка достали случайную карту и положили на стол случайной стороной вниз. Какова вероятность того, что нижняя сторона карты на столе белая, при условии, что ее верхняя сторона красная?

\z{} Четверо идут в поход, у них есть 4-местная палатка, 2 одноместных спальника и один двуместный. Идут дожди и поэтому на крайних местах в палатке лежать мокро и плохо. Парни из двуместного спальника предлагают: давайте двуместный спальник будет в середине с вероятностью $\frac12$ (определяется подкидыванием монеты), а места между людьми в большом спальнике и двумя в одноместных спальниках также распределяются подкидыванием монеты.
\textbf{а)} Какова вероятность спать у края для парня из двуместного спальника?
\textbf{б)} Через некоторое время появляется более честное предложение: давайте (т. к. спальников всего три) большой спальник будет внутри с вероятностью $\frac13$. Посчитайте для каждого вероятности спать у края в этом случае.
\textbf{в)} Предложите как справедливо распределять места.

\z{} В жюри из трёх человек вердикт выносят большинством голосов. Председатель и помощник председателя принимают верное решение независимо с вероятностями 0.7 и 0.9, а третий для этого бросает монету. На сколько изменится у жюри вероятность вынести верное решение, если третий начнет копировать решение председателя?

\z{} Ваня хочет выиграть две партии в теннис подряд из трех, играя поочередно с двумя игроками. С кем лучше играть первую партию: с более сильным игроком или с более слабым?

\z{} Заключённые X, Y и Z приговорены к смертной казни. Губернатор случайным образом выбирает одного из них и милует его. Стражник знает, кто помилован, но не имеет права сказать этого. X просит стражника сказать ему имя другого заключённого, которого казнят (если казнят обоих, ответ стражника определяется броском монеты). Стражник говорит X, что казнят Y. X рад это слышать, т. к. считает, что с учетом этой информации его вероятность выживания теперь 1/2. X рассказывает все Z, который тоже обрадовался полагая, что теперь выживает с вероятностью 2/3. Кто-нибудь из них прав в этой ситуации?

\end{document}
