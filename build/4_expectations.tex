
\documentclass[11pt,a4paper]{article}
%\usepackage{russ}
\usepackage[utf8]{inputenc}
\usepackage[english,russian]{babel}

\usepackage{amsmath, amssymb}
\usepackage{amscd}
\usepackage{amsthm}
\usepackage[left=3cm,right=3cm,top=3cm,bottom=1cm]{geometry}
\usepackage{geometry}
\usepackage{fancyhdr}
\usepackage{color}
\usepackage[pdftex,dvipsnames]{xcolor}  % Coloured text etc.
\usepackage{graphicx}
\usepackage{xargs}                      % Use more than one optional parameter in a new 
\graphicspath{ {./pic/} }

\def\Z{\mathbb{Z}}
\def\N{\mathbb{N}}
\def\R{\mathbb{R}}
\def\F{\mathbb{F}}
\def\Q{\mathbb{Q}}

\def\E{\mathrm{\mathbb{E}}}
\def\D{\mathrm{\mathbb{D}}}
\def\P{\mathrm{P}}

\newcounter{znum}
\newcommand{\z}[1]{\addtocounter{znum}{1} \textbf{Задача \arabic{znum}#1. }}

\newcounter{defnum}
\newcommand{\df}[1]{\addtocounter{defnum}{1} \textbf{Определение \arabic{defnum}.} {\it #1}}

\renewcommand{\!}{\textcolor{red}{!}}

\usepackage{wrapfig}
\usepackage{graphicx} % Allows including images
\usepackage{float}


\usepackage[
	% disable,
	colorinlistoftodos, prependcaption, textsize=normalsize,
	linecolor=Dandelion, backgroundcolor=Yellow!25, bordercolor=Dandelion,
	textwidth=100
]{todonotes}

\newcommandx{\note}[2][1=]{\todo[size=footnotesize, #1]{#2}}
\newcommandx{\longnote}[2][1=]{\todo[inline, nolist, linecolor=Gray, backgroundcolor=Gray!25, bordercolor=Gray, #1]{#2}}
\newcommandx{\problem}[2][1=]{\todo[backgroundcolor=red!25, linecolor=red, bordercolor=red, #1]{#2}}
\newcommandx{\question}[2][1=]{\todo[
	noprepend, inline,
	linecolor=RoyalBlue, backgroundcolor=RoyalBlue!25, bordercolor=RoyalBlue, #1]{#2}
}
\newcommandx{\thought}[2][1=]{\todo[
	noprepend, inline,
	linecolor=Orchid, backgroundcolor=Orchid!25, bordercolor=Orchid, #1]{#2}
}

\newcommandx{\checkref}[2][1={check reference validity}, 2=]{
	\todo[linecolor=red, backgroundcolor=red!25, bordercolor=red, size=footnotesize, #2]{#1}
}
\newcommandx{\needref}[2][1={need reference}, 2=]{
	\todo[linecolor=red, backgroundcolor=red!25, bordercolor=red, size=footnotesize, #2]{#1}
}
\newcommand{\EMPH}[1]{{\large \textit{\textbf{#1}}}}


\begin{document}

\pagestyle{empty}

\begin{center}
	\large
	\textbf{Матожидания.}\\
\end{center}


Пусть дано вероятностное пространство $\Omega$. \emph{Случайной величиной} (на вероятностном пространстве $\Omega$) называется произвольная функция $\xi$ из $\Omega$ в вещественные числа\footnote{Традиционно в олдскульном тервере для обозначения случайных величин используются станные греческие буквы вроде $\xi, \zeta, \eta$.}. \emph{Математическим ожиданием} (оно же \emph{матожидание}) случайной величины $\xi$ называется число\footnote{реже то же самое часто обозначают $M\xi$.}
$$\E\xi = \sum_{\omega \in \Omega} \xi(\omega) P(\omega).$$

\z{} Чему равно матожидание числа, выпадающего на игральном кубике?

\z{} 
\textbf{а)} Докажите, что:
$$\E\xi = \sum_{y \in \xi(\Omega)} \, y \cdot P(\xi = y)$$
($\xi(\Omega)$ обозначает множество всех значений функции $\xi$ на множестве $\Omega$).

\textbf{б)} Предположим случайная величина $\xi$ принимает только неотрицательные целые значения. Докажите, что:
$$\E\xi = \sum_{i=0}^\infty P(\xi > i).$$

\z{} В Китае ввели закон, имеющий целью уменьшить прирост населения. Если в семье первый ребёнок — мальчик, семье не разрешается больше иметь детей. Если же первый ребёнок — девочка, можно завести еще одного ребёнка. Как выполнение закона повлияет на соотношение численностей мужского и женского населения в Китае (вероятностью рождения двойняшек пренебречь).

\z{} Докажите, что $\E(\xi + \zeta) = \E \xi + \E \zeta$.

\vspace{0.2cm}

На предыдущей задаче основан важный \emph{принцип линейности матожидания}\footnote{Эта техника подсчета матожиданий, подробно описанна в этой (возможно, уже вам известной) книге: "Кормен Т., Лейзерсон Ч., Ривест Р., Штайн K. - Алгоритмы: построение и анализ" (глава 5).}:
''Пусть вам дали сложную и страшную случайную величину $\xi$ и требуют посчитать ее матожидание. Но вы вы догадались, что ее можно разложить в сумму некоторого числа более понятных величин: $\xi = \xi_1 + \xi_2 + \ldots + \xi_N$. Тогда $E\xi = E\xi_1 + E\xi_2 + \ldots + E\xi_N$.'' Вот вам три задачи, чтобы распробовать, как этот принцип применять:

\vspace{0.2cm}

\z{} Каждый из $n$ людей положил в мешок по подарку, затем подарки в мешке перемешали и каждый вытащил подарок для себя. Найдите матожидание числа тех, кому достался подарок, который они сами принесли.

\z{} Собралось k случайных людей. Найдите матожидание числа пар людей с совпадающими днями рождения (предполагается, что никто не родился 29 февраля).

\z{} Если человек тратит в очереди минуту на ожидание, будем говорить, что бесцельно затрачена одна человеко-минута. В очереди в банке стоит восемь человек, из них пятеро планируют простые операции, занимающие 1 минуту, а остальные планируют операции, занимающие 5 минут. Рассмотрим суммарное количество бесцельно затраченных человеко-минут, найдите его:
\textbf{а)} наименьшее и наибольшее возможное значения;
\textbf{б)} математическое ожидание, при условии, что клиенты встают в очередь в случайном порядке.

\vspace{0.5cm}

Пусть $A$ -- событие с ненулевой вероятностью. \emph{Условным математическим ожиданием} случайной величины $\xi$ при условии $A$ называется число
$$\E(\xi | A) = \frac{1}{P(A)} \sum_{\omega \in A} \xi(\omega) P(\omega).$$

\z{} Докажите, что:
$$ \E(\xi | A) = \sum_{y \in \xi(\Omega)} y \cdot P(\xi = y | A)$$

\z{}
\textbf{а)} Кинули два кубика, каково матожидание числа выпавшего на первом кубике при условии, что в сумме выпало 10?\\
\textbf{б)} (пример \emph{регрессии к среднему}) Предположим, что люди рождаются с IQ принимающим целые значения из $[50, 150]$ с равными вероятностями. При измерении IQ возникает ошибка, равновероятно принимающая целые значения из $[-50, 50]$ (ошибка независима с IQ). Каково матожидание IQ человека, написавшего тест на $100 + k$ баллов.


\begin{center}
	\large
	\textbf{Неравенства и закон больших чисел.}
\end{center}

\z{ (неравенство Маркова)} Пусть $X$ - неотрицательная случайная величина (т. е. её значения на любом элементарном исходе неотрицательны). Докажите что $\P(X \geqslant a) \leqslant \frac{\E(X)}{a}$ (где $a$ -- любое положительное число).

\z{} В некоторой лотерее билет стоит 100 рублей и 40\% средств идут на выплату призов. Докажите что вероятность выиграть 5000 рублей меньше 1\%.

\df{Случайные величины $X$ и $Y$ называются \textbf{независимыми}, если любые двух $a$ и $b$ события вида $X = a$ и $Y = b$ независимы.}

\z{} Пусть $A$ и $B$ -- произвольные множества в $\R$. Докажите, что если случайные величины $X$ и $Y$ независимы, то события $X \in A$ и $Y \in B$ независимы.

\z{} Приведите пример двух независимых и двух зависимых случайных величин.

\z{} Докажите, что если случайные величины $X$ и $Y$ независимы, то $\E (X\cdot Y) = \E(X)\cdot \E(Y)$.

\z{} Приведите контрпример к утверждению предыдущей задачи для зависимых случайных величин.

\df{\textbf{Дисперсией}\footnote{Также это называют вариацией и обозначают $Var(X)$.} случайной величины $X$ называется число}
$$\D(X) = \E(X - \E(X))^2$$.

\z{} Докажите, что $\D (X) = \E X^2 - (\E X)^2$.

\z{} Дисперсия величины $X$ равна $d$, чему тогда равна дисперсия

\textbf{а)} величины $X + c$?

\textbf{б)} величины $a \cdot X$?

\z{} Докажите, что если случайные величины $X$ и $Y$ независимы, то $\D (X + Y) = \D(X) + \D(Y)$.

\z{ (неравенство Чебышева)} Докажите что $\P(| X - \E(X)| \geqslant a) \leqslant \frac{\D(X)}{a^2}$.

\z{} Пусть $X_1, \ldots, X_n$ -- независимые случайные величины с дисперсией $d$. Найдите дисперсию величины $S_n = \frac{X_1 + \ldots + X_n}{n}$

\z{ (Слабый закон больших чисел)} Пусть $X_1, \ldots, X_n$ -- независимые случайные величины с неизвестным вам матожиданием $m$. Известно, что дисперсия каждой из них не превосходит $d$. Вам дали задание оценить $m$ с точностью $\varepsilon > 0$ и дали право ошибаться с вероятностью $\delta > 0$. Какое нужно взять $n$, чтобы оценка $S_n = \frac{X_1 + \ldots + X_n}{n}$ подходила?


\end{document}
