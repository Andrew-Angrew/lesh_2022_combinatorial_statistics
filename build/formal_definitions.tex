
\documentclass[11pt,a4paper]{article}
%\usepackage{russ}
\usepackage[utf8]{inputenc}
\usepackage[english,russian]{babel}

\usepackage{amsmath, amssymb}
\usepackage{amscd}
\usepackage{amsthm}
% \usepackage[left=1.5cm,right=1.5cm,top=1cm,bottom=1cm]{geometry}
\usepackage[top=3cm]{geometry}
\usepackage{geometry}
\usepackage{fancyhdr}
\usepackage{color}
\usepackage[pdftex,dvipsnames]{xcolor}  % Coloured text etc.
\usepackage{graphicx}
\usepackage{xargs}                      % Use more than one optional parameter in a new 
\graphicspath{ {./pic/} }

\def\Z{\mathbb{Z}}
\def\N{\mathbb{N}}
\def\R{\mathbb{R}}
\def\F{\mathbb{F}}
\def\Q{\mathbb{Q}}

\def\E{\mathrm{\mathbb{E}}}
\def\D{\mathrm{\mathbb{D}}}
\def\P{\mathrm{P}}

\newcounter{znum}
\newcommand{\z}[1]{\addtocounter{znum}{1} \textbf{Задача \arabic{znum}#1. }}

\newcounter{defnum}
\newcommand{\df}[1]{\addtocounter{defnum}{1} \textbf{Определение \arabic{defnum}.} {\it #1}}

\renewcommand{\!}{\textcolor{red}{!}}

\usepackage{wrapfig}
\usepackage{graphicx} % Allows including images
\usepackage{float}


\usepackage[
	% disable,
	colorinlistoftodos, prependcaption, textsize=normalsize,
	linecolor=Dandelion, backgroundcolor=Yellow!25, bordercolor=Dandelion,
	textwidth=100
]{todonotes}

\newcommandx{\note}[2][1=]{\todo[size=footnotesize, #1]{#2}}
\newcommandx{\longnote}[2][1=]{\todo[inline, nolist, linecolor=Gray, backgroundcolor=Gray!25, bordercolor=Gray, #1]{#2}}
\newcommandx{\problem}[2][1=]{\todo[backgroundcolor=red!25, linecolor=red, bordercolor=red, #1]{#2}}
\newcommandx{\question}[2][1=]{\todo[
	noprepend, inline,
	linecolor=RoyalBlue, backgroundcolor=RoyalBlue!25, bordercolor=RoyalBlue, #1]{#2}
}
\newcommandx{\thought}[2][1=]{\todo[
	noprepend, inline,
	linecolor=Orchid, backgroundcolor=Orchid!25, bordercolor=Orchid, #1]{#2}
}

\newcommandx{\checkref}[2][1={check reference validity}, 2=]{
	\todo[linecolor=red, backgroundcolor=red!25, bordercolor=red, size=footnotesize, #2]{#1}
}
\newcommandx{\needref}[2][1={need reference}, 2=]{
	\todo[linecolor=red, backgroundcolor=red!25, bordercolor=red, size=footnotesize, #2]{#1}
}
\newcommand{\EMPH}[1]{{\large \textit{\textbf{#1}}}}

\newcommand{\double}[1]{
	#1
	\vspace{2cm}
	#1
}



\begin{document}

\thispagestyle{empty}

\double{

\begin{center}
	\large
	\textbf{Вероятностные пространства}
	\normalsize
\end{center}

Говорят, что конечное множество $\Omega$ является \emph{конечным вероятностным пространством}, если для любого его подмножества $A \subset \Omega$ задана его \emph{вероятность} $P(A)$ и выполняются следующие условия:

$\bullet$ $P(\varnothing) = 0, P(\Omega) = 1$.

$\bullet$ Для любого $A \subset \Omega$ верно $P(A) \geqslant 0$.

$\bullet$ Для любых $A, B\subset \Omega$ верно $P(A\cup B) = P(A) + P(B) - P(A\cap B)$.

Элементы $\omega \in \Omega$ принято называть \emph{элементарными исходами}, а подмножества $\Omega$ --- \emph{событиями}. События A и B называются \emph{независимыми}, если $P(A \cap B) = P(A) \cdot P(B)$.

\z{} Придумайте вероятностные пространства для задач 1 (про 2 монеты) и 3 (про 3 коробки с одним призом) из затравочного листка.

\z{} $A$ и $B$ независимы. Докажите что $A$ и $\overline{B} = \Omega \backslash B$ независимы.

\z{} Кинули два игральных кубика. Независимы ли следующие события:

\textbf{а)} «на первом выпала тройка» и «выпавшая сумма чётна»;

\textbf{б)} «на первом выпала тройка» и «выпавшая сумма больше 6».


Если $P(B) \ne 0$, то \emph{условной вероятностью} события A при условии события~B называется число $P(A|B) = P(A \cap B)/P(B)$.

\z{} Пусть $P(B) \ne 0$, тогда A и B независимы $ \Leftrightarrow P(A) = P(A|B)$.

\z{ (формула Байеса)} Проверьте что $\P(B | A) = \frac{\P(A | B)\P(B)}{\P(A)}$.

}

\end{document}
