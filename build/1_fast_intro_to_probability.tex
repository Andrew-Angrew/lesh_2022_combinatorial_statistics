
\documentclass[11pt,a4paper]{article}
%\usepackage{russ}
\usepackage[utf8]{inputenc}
\usepackage[english,russian]{babel}

\usepackage{amsmath, amssymb}
\usepackage{amscd}
\usepackage{amsthm}
% \usepackage[left=1.5cm,right=1.5cm,top=1cm,bottom=1cm]{geometry}
\usepackage[top=3cm]{geometry}
\usepackage{geometry}
\usepackage{fancyhdr}
\usepackage{color}
\usepackage[pdftex,dvipsnames]{xcolor}  % Coloured text etc.
\usepackage{graphicx}
\usepackage{xargs}                      % Use more than one optional parameter in a new 
\graphicspath{ {./pic/} }

\def\Z{\mathbb{Z}}
\def\N{\mathbb{N}}
\def\R{\mathbb{R}}
\def\F{\mathbb{F}}
\def\Q{\mathbb{Q}}

\def\E{\mathrm{\mathbb{E}}}
\def\D{\mathrm{\mathbb{D}}}
\def\P{\mathrm{P}}

\newcounter{znum}
\newcommand{\z}[1]{\addtocounter{znum}{1} \textbf{Задача \arabic{znum}#1. }}

\newcounter{defnum}
\newcommand{\df}[1]{\addtocounter{defnum}{1} \textbf{Определение \arabic{defnum}.} {\it #1}}

\renewcommand{\!}{\textcolor{red}{!}}

\usepackage{wrapfig}
\usepackage{graphicx} % Allows including images
\usepackage{float}


\begin{document}
\pagestyle{empty}


\begin{center}
	\large
	\textbf{Теория}
	\normalsize
\end{center}

Говорят, что конечное множество $\Omega$ является \emph{конечным вероятностным пространством}, если для любого его подмножества $A \subset \Omega$ задана его \emph{вероятность} $P(A)$ и выполняются следующие условия:

$\bullet$ $P(\varnothing) = 0, P(\Omega) = 1$.

$\bullet$ Для любого $A \subset \Omega$ верно $P(A) \geqslant 0$.

$\bullet$ Для любых $A, B\subset \Omega$ верно $P(A\cup B) = P(A) + P(B) - P(A\cap B)$.

Элементы $\omega \in \Omega$ принято называть \emph{элементарными исходами}, а подмножества $\Omega$ --- \emph{событиями}. События A и B называются \emph{независимыми}, если $P(A \cap B) = P(A) \cdot P(B)$.

\z{} Придумайте вероятностные пространства для задач 1 (про 2 монеты) и 2 (про 2 кубика) из затравочного листка.

\z{} $A$ и $B$ независимы. Докажите что $A$ и $\overline{B} = \Omega \backslash B$ независимы.

\vspace{0.3cm}

Если $P(B) \ne 0$, то \emph{условной вероятностью} события A при условии события~B называется число $P(A|B) = P(A \cap B)/P(B)$.

\vspace{0.3cm}

\z{} Пусть $P(B) \ne 0$, тогда A и B независимы $ \Leftrightarrow P(A) = P(A|B)$.

\z{ (формула Байеса)} Проверьте что $\P(B | A) = \frac{\P(A | B)\P(B)}{\P(A)}$.

\begin{center}
	\large
	\textbf{Задачи}
	\normalsize
\end{center}

\z{} Одновременно подбрасываются 2 монеты, какова вероятность того что выпадет хотя бы один орел?

\z{} Кинули два игральных кубика. С какой вероятностью \textbf{а)} сумма выпавших чисел равна 9? а 10? \textbf{б)} на втором кубике выпало больше, чем на первом?\\
Независимы ли следующие события: \textbf{в)} «на первом выпала тройка» и «выпавшая сумма чётна»; \textbf{г)} «на первом выпала тройка» и «выпавшая сумма больше 6».

\z{} В 90\% ресторанов, где есть хорошие бургеры, нет хороших крылышек, а в 90\% ресторанов, где есть хорошие крылышки, нет хороших бургеров. Значит ли это, наличия хороших бургеров и хороших крылышек являются зависимыми событиями?

\z{} По результатам исследования среди выздоровевших людей 12\% принимали лекарство, а среди не выздоровевших -- 1\%. Журналисты сделали два вывода: 1) Для тех, кто принимает лекарство, вероятность выздороветь в 12 раз больше, чем вероятность не выздороветь. 2) При условии приема лекарства вероятность не выздороветь равна 1\%, следовательно вероятность выздороветь равна 99\%. \textbf{а)} Какие из выводов журналистов верны?
\textbf{б)} Как посчитать вероятность выздороветь при условии приема лекарства и каких данных для этого не хватает?

\z{} В одной лотерее у каждого игрока есть два набора из 15 различных чисел от 1 до 90. Игрок выигрывает джекпот, если случайный набор из 15 различных чисел от 1 до 90 (все такие наборы равновероятны) совпадет с одним из его наборов. Сравните с $1/2$ вероятность того, что кто-то получит джекпот, если 10 миллиардов человек будут играть в эту лотерею каждый день в течении 1000 лет (разрешается использовать внешний неорганический мозг).


\end{document}
