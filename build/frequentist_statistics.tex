
\documentclass[11pt,a4paper]{article}
%\usepackage{russ}
\usepackage[utf8]{inputenc}
\usepackage[english,russian]{babel}

\usepackage{amsmath, amssymb}
\usepackage{amscd}
\usepackage{amsthm}
% \usepackage[left=1.5cm,right=1.5cm,top=1cm,bottom=1cm]{geometry}
\usepackage{geometry}
\usepackage{fancyhdr}
\usepackage{color}
\usepackage[pdftex,dvipsnames]{xcolor}  % Coloured text etc.
\usepackage{graphicx}
\usepackage{xargs}                      % Use more than one optional parameter in a new 
\graphicspath{ {./pic/} }

\def\Z{\mathbb{Z}}
\def\N{\mathbb{N}}
\def\R{\mathbb{R}}
\def\F{\mathbb{F}}
\def\Q{\mathbb{Q}}

\def\E{\mathrm{\mathbb{E}}}
\def\D{\mathrm{\mathbb{D}}}
\def\P{\mathrm{P}}

\newcounter{znum}
\newcommand{\z}[1]{\addtocounter{znum}{1} \textbf{Задача \arabic{znum}#1. }}

\newcounter{defnum}
\newcommand{\df}[1]{\addtocounter{defnum}{1} \textbf{Определение \arabic{defnum}.} {\it #1}}

\renewcommand{\!}{\textcolor{red}{!}}

\usepackage{wrapfig}
\usepackage{graphicx} % Allows including images
\usepackage{float}


\usepackage[
	% disable,
	colorinlistoftodos, prependcaption, textsize=normalsize,
	linecolor=Dandelion, backgroundcolor=Yellow!25, bordercolor=Dandelion,
	textwidth=100
]{todonotes}

\newcommandx{\note}[2][1=]{\todo[size=footnotesize, #1]{#2}}
\newcommandx{\longnote}[2][1=]{\todo[inline, nolist, linecolor=Gray, backgroundcolor=Gray!25, bordercolor=Gray, #1]{#2}}
\newcommandx{\problem}[2][1=]{\todo[backgroundcolor=red!25, linecolor=red, bordercolor=red, #1]{#2}}
\newcommand{\EMPH}[1]{{\large \textit{\textbf{#1}}}}

\begin{document}
\pagestyle{empty}

\begin{center}
	\large
	\textbf{Байесовская статистика.}\\
\end{center}

\z{} Известно, что синдрому внезапной смерти подвержен один человек из 10000. Тест на этот синдром дает верный результат с вероятностью 99\%. Тест показывает, что у вас синдром внезапной смерти. С какой вероятностью это действительно так?

\vspace{0.1cm}
\textbf{А теперь кидаем монеты 100500 раз...}
\vspace{0.1cm}

\z{} Монету бросили 10 раз. Какова вероятность того, что \textbf{а)}~все 10 раз выпал орёл \textbf{б)}~сначала выпало 5 орлов, а затем 5 решек \textbf{в)}~выпало 5 орлов и 5 решек (в произвольном порядке)?

\z{} Монету бросили $n$ раз. С какой вероятностью \textbf{а)}~орёл выпал $k$ раз? \textbf{б)}~а если монетка была ''нечестная'' и орёл выпадает с вероятностью $p$, а не $1/2$?

\z{ (!)} Из 100 монет одна фальшивая имеет два орла. Выбрали случайно монету, бросили 5 раз: выпали все орлы. С какой вероятностью, если её бросить ещё 10 раз, снова выпадут все орлы?

\z{} Вы приходите в бар на диком западе играть в некоторую полностью случайную азартную игру. Вероятность победы в ней равна $1/2$. Известно, что 10\% игроков на диком западе -- шулеры, которые выигрывают у честных игроков с вероятностью 90\%. Вы играете честно. Какова вероятность, что вы играете с шулером, если \textbf{а)} 

\z{} У шулера в кармане две монетки одна честная, другая нечестная, выпадающая орлом чаще, чем честная. Он достает какую-то из них (какую захочет) и кидает 10 раз. Выпадает 10 орлов. С какой вероятностью взятая им монетка нечестная?


\begin{center}
	\large
	\textbf{Частотная статистика.}\\
\end{center}

\z{ (!)} Расскажите мне все про тест точный тест Фишера: в чем состоит нулевая гипотеза; как считать pvalue; докажите корректность.

\z{ (!)} Расскажите мне все про тест Манна-Уитни: в чем состоит нулевая гипотеза; как считать pvalue; докажите корректность.

\z{} Придумайте, как посчитать статистику Манна-Уитни за $O(n\log n)$ (сортировка + O(n))

\z{} Приведите пример таких величин X, Y что $\P(Y > X) > 1/2$, но $\E(Y) < \E(X)$.

\z{} Приведите пример таких величин X, Y и Z, что 
$$\P(Y > X) > 1/2,\quad  \P(Z > Y) > 1/2 \quad \mbox{и} \quad \P(X > Z) > 1/2.$$

\z{*} Посчитайте дисперсию статистики Манна-Уитни при условии нулевой гипотезы.

\z{**} Посчитайте правильное pvalue для "теста Гальтона".

\end{document}
