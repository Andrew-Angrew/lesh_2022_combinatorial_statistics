
\documentclass[11pt,a4paper]{article}
%\usepackage{russ}
\usepackage[utf8]{inputenc}
\usepackage[english,russian]{babel}

\usepackage{amsmath, amssymb}
\usepackage{amscd}
\usepackage{amsthm}
%\usepackage[left=1.5cm,right=1.5cm,top=1cm,bottom=1cm]{geometry}
\usepackage{geometry}
\usepackage{fancyhdr}
\usepackage{color}
\usepackage[pdftex,dvipsnames]{xcolor}  % Coloured text etc.
\usepackage{graphicx}
\usepackage{xargs}                      % Use more than one optional parameter in a new 
\graphicspath{ {./pic/} }

\def\Z{\mathbb{Z}}
\def\N{\mathbb{N}}
\def\R{\mathbb{R}}
\def\F{\mathbb{F}}
\def\Q{\mathbb{Q}}

\def\E{\mathrm{\mathbb{E}}}
\def\D{\mathrm{\mathbb{D}}}
\def\P{\mathrm{P}}

\newcounter{znum}
\newcommand{\z}[1]{\addtocounter{znum}{1} \textbf{Задача \arabic{znum}#1. }}

\newcounter{defnum}
\newcommand{\df}[1]{\addtocounter{defnum}{1} \textbf{Определение \arabic{defnum}.} {\it #1}}

\renewcommand{\!}{\textcolor{red}{!}}

\usepackage{wrapfig}
\usepackage{graphicx} % Allows including images
\usepackage{float}


\usepackage[
	% disable,
	colorinlistoftodos, prependcaption, textsize=normalsize,
	linecolor=Dandelion, backgroundcolor=Yellow!25, bordercolor=Dandelion,
	textwidth=100
]{todonotes}

\newcommandx{\note}[2][1=]{\todo[size=footnotesize, #1]{#2}}
\newcommandx{\longnote}[2][1=]{\todo[inline, nolist, linecolor=Gray, backgroundcolor=Gray!25, bordercolor=Gray, #1]{#2}}
\newcommandx{\problem}[2][1=]{\todo[backgroundcolor=red!25, linecolor=red, bordercolor=red, #1]{#2}}
\newcommandx{\question}[2][1=]{\todo[
	noprepend, inline,
	linecolor=RoyalBlue, backgroundcolor=RoyalBlue!25, bordercolor=RoyalBlue, #1]{#2}
}
\newcommandx{\thought}[2][1=]{\todo[
	noprepend, inline,
	linecolor=Orchid, backgroundcolor=Orchid!25, bordercolor=Orchid, #1]{#2}
}

\newcommandx{\checkref}[2][1={check reference validity}, 2=]{
	\todo[linecolor=red, backgroundcolor=red!25, bordercolor=red, size=footnotesize, #2]{#1}
}
\newcommandx{\needref}[2][1={need reference}, 2=]{
	\todo[linecolor=red, backgroundcolor=red!25, bordercolor=red, size=footnotesize, #2]{#1}
}
\newcommand{\EMPH}[1]{{\large \textit{\textbf{#1}}}}


\begin{document}

\thispagestyle{empty}
\begin{center}
	\large
	\textbf{Еще задачи}
	\normalsize
\end{center}

\z{} Монету бросили 10 раз. Какова вероятность того, что \textbf{а)}~все 10 раз выпал орёл \textbf{б)}~сначала выпало 5 орлов, а затем 5 решек \textbf{в)}~выпало 5 орлов и 5 решек (в произвольном порядке)?

\z{ (!)} Из 100 монет одна фальшивая имеет два орла. Выбрали случайно монету, бросили 5 раз: выпали все орлы. С какой вероятностью, если её бросить ещё 10 раз, снова выпадут все орлы?

\z{} Будем считать, что рождение девочки и мальчика равновероятны. Известно, что в некоторой семье двое детей. \textbf{а)} Какова вероятность того, что из них один мальчик и одна девочка? А если известно что
\textbf{б)} один из детей — мальчик.
\textbf{в)} Старший ребенок - мальчик.
\textbf{г)} Один из детей мальчик, родившийся в понедельник.

\z{} В контрольной 3 задачи. Ученик, подготовившийся к контрольной, решает каждую из
них с вероятностью $4/5$, а не готовившийся --- с вероятностью $1/5$. Известно что он готовится к контрольной с вероятностью $3/4$ и не готовится с вероятностью $1/4$. Чему равна условная вероятность того, что он готовился к контрольной, при условии, что он решил \textbf{а)} все задачи \textbf{б)} две задачи?

\z{} Известно, что синдрому внезапной смерти подвержен один человек из 10000. Тест на этот синдром дает верный результат с вероятностью 99\%. Тест показывает, что у вас синдром внезапной смерти. С какой вероятностью это действительно так?

\z{} Вероятность рождения двойняшек в Швамбрании равна $p$, тройняшки в Швамбрании не рождаются.
Оцените вероятность того, что встреченный на улице швамбранец – один из пары двойняшек?

\z{ (!?)} В одной лотерее у каждого игрока есть два набора из 15 различных чисел от 1 до 90. Игрок выигрывает джекпот, если случайный набор из 15 различных чисел от 1 до 90 (все такие наборы равновероятны) совпадет с одним из его наборов. Сравните с $1/2$ вероятность того, что кто-то получит джекпот, если 10 миллиардов человек будут играть в эту лотерею каждый день в течении 1000 лет (разрешается использовать внешний неорганический мозг).

\z{} \textbf{а)} Двое договорились играть в орлянку до определенного количества побед. Тот, кто первым достиг этого количества побед, забирает денежный приз. Но игра была прервана, когда первому игроку осталось одержать 3 победы, а второму - одну. В каком отношении должен быть поделен приз (так, чтобы каждому досталось количество денег, пропорциональное вероятности его выигрыша)?
\textbf{б*)} А если первому осталось n побед, а второму - k побед?

\z{} Все жители планеты $X$ опубликовали по $n$ дней, когда они будут ходить в гости, и по $k$ дней, когда они будут принимать гостей. Оказалось что все жители $X$ могут погостить у всех в соответствии с их расписаниями.
\textbf{а)} С какой вероятностью все "походные" дни $i$-того жителя будут предшествовать всем его "приемным дням" в случайной перестановке дней $X$-ианского года?
\textbf{б)} Докажите что на $X$ не более чем $C_{n+k}^k$ жителей.


\end{document}
