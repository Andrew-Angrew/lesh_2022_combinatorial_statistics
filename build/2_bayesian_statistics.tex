
\documentclass[11pt,a4paper]{article}
%\usepackage{russ}
\usepackage[utf8]{inputenc}
\usepackage[english,russian]{babel}

\usepackage{amsmath, amssymb}
\usepackage{amscd}
\usepackage{amsthm}
% \usepackage[left=1.5cm,right=1.5cm,top=1cm,bottom=1cm]{geometry}
\usepackage{geometry}
\usepackage{fancyhdr}
\usepackage{color}
\usepackage[pdftex,dvipsnames]{xcolor}  % Coloured text etc.
\usepackage{graphicx}
\usepackage{xargs}                      % Use more than one optional parameter in a new 
\graphicspath{ {./pic/} }

\def\Z{\mathbb{Z}}
\def\N{\mathbb{N}}
\def\R{\mathbb{R}}
\def\F{\mathbb{F}}
\def\Q{\mathbb{Q}}

\def\E{\mathrm{\mathbb{E}}}
\def\D{\mathrm{\mathbb{D}}}
\def\P{\mathrm{P}}

\newcounter{znum}
\newcommand{\z}[1]{\addtocounter{znum}{1} \textbf{Задача \arabic{znum}#1. }}

\newcounter{defnum}
\newcommand{\df}[1]{\addtocounter{defnum}{1} \textbf{Определение \arabic{defnum}.} {\it #1}}

\renewcommand{\!}{\textcolor{red}{!}}

\usepackage{wrapfig}
\usepackage{graphicx} % Allows including images
\usepackage{float}


\usepackage[
	% disable,
	colorinlistoftodos, prependcaption, textsize=normalsize,
	linecolor=Dandelion, backgroundcolor=Yellow!25, bordercolor=Dandelion,
	textwidth=100
]{todonotes}

\newcommandx{\note}[2][1=]{\todo[size=footnotesize, #1]{#2}}
\newcommandx{\longnote}[2][1=]{\todo[inline, nolist, linecolor=Gray, backgroundcolor=Gray!25, bordercolor=Gray, #1]{#2}}
\newcommandx{\problem}[2][1=]{\todo[backgroundcolor=red!25, linecolor=red, bordercolor=red, #1]{#2}}
\newcommand{\EMPH}[1]{{\large \textit{\textbf{#1}}}}

\begin{document}
\pagestyle{empty}

\begin{center}
	\Large
	\textbf{Байесовская статистика.}\\
\end{center}

\begin{center}
	\large
	\textbf{Условные вероятности.}\\
\end{center}


\z{} В мешке есть три карты. У одной обе стороны белые, у другой --- красные, а у третьей --- одна белая и одна красная. Из мешка достали случайную карту и положили на стол случайной стороной вниз. Какова вероятность того, что нижняя сторона карты на столе белая, при условии, что ее верхняя сторона красная?

\z{} Известно, что синдрому внезапной смерти подвержен один человек из 10000. Тест на этот синдром дает верный результат с вероятностью 99\%. Тест показывает, что у вас синдром внезапной смерти. С какой вероятностью это действительно так?

\z{} В контрольной 3 задачи. Ученик, подготовившийся к контрольной, решает каждую из
них с вероятностью $4/5$, а не готовившийся --- с вероятностью $1/5$. Известно что он готовится к контрольной с вероятностью $3/4$ и не готовится с вероятностью $1/4$. Чему равна условная вероятность того, что он готовился к контрольной, при условии, что он решил \textbf{а)} все задачи \textbf{б)} две задачи?

\z{} Будем считать, что рождение девочки и мальчика равновероятны. Известно, что в некоторой семье двое детей. \textbf{а)} Какова вероятность того, что из них один мальчик и одна девочка? А если известно что
\textbf{б)} один из детей — мальчик.
\textbf{в)} Старший ребенок - мальчик.
\textbf{г)} Один из детей мальчик, родившийся в понедельник.


\begin{center}
	\large
	\textbf{Кидаем монеты 100500 раз...}\\
\end{center}

\z{} Монету бросили 10 раз. Какова вероятность того, что \textbf{а)}~все 10 раз выпал орёл \textbf{б)}~сначала выпало 5 орлов, а затем 5 решек \textbf{в)}~выпало 5 орлов и 5 решек (в произвольном порядке)?

\z{} Монету бросили $n$ раз. С какой вероятностью \textbf{а)}~орёл выпал $k$ раз? \textbf{б)}~а если монетка была ''нечестная'' и орёл выпадает с вероятностью $p$, а не $1/2$?

\z{} Из 100 монет одна фальшивая имеет два орла. Выбрали случайно монету, бросили 5 раз: выпали все орлы. \textbf{а)}~С какой вероятностью была выбрана фальшивая монета? \textbf{б)}~С какой вероятностью, если её бросить ещё 10 раз, снова выпадут все орлы?

\z{} Вы приходите в бар на диком западе играть в некоторую полностью случайную азартную игру. Вероятность победы в ней равна $1/2$. Известно, что 10\% игроков на диком западе -- шулеры, которые выигрывают у честных игроков с вероятностью 90\%. Вы играете честно. Вы сыграли 10 игр, какова вероятность, что вы играете с шулером, если вы проиграли \textbf{а)}~8 раз \textbf{б)}~9 раз? Попробуйте ради интереса сначала оценить эти вероятности интуитивно.

\z{} У шулера в кармане две монетки одна честная, другая нечестная, выпадающая орлом чаще, чем честная. Он достает какую-то из них (какую захочет) и кидает 10 раз. Выпадает 10 орлов. С какой вероятностью взятая им монетка нечестная?

\end{document}
