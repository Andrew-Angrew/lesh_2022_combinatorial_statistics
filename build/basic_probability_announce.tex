
\documentclass[11pt,a4paper]{article}
%\usepackage{russ}
\usepackage[utf8]{inputenc}
\usepackage[english,russian]{babel}

\usepackage{amsmath, amssymb}
\usepackage{amscd}
\usepackage{amsthm}
% \usepackage[left=1.5cm,right=1.5cm,top=1cm,bottom=1cm]{geometry}
\usepackage[right=4cm]{geometry}
\usepackage{geometry}
\usepackage{fancyhdr}
\usepackage{color}
\usepackage[pdftex,dvipsnames]{xcolor}  % Coloured text etc.
\usepackage{graphicx}
\usepackage{xargs}                      % Use more than one optional parameter in a new 
\graphicspath{ {./pic/} }

\def\Z{\mathbb{Z}}
\def\N{\mathbb{N}}
\def\R{\mathbb{R}}
\def\F{\mathbb{F}}
\def\Q{\mathbb{Q}}

\def\E{\mathrm{\mathbb{E}}}
\def\D{\mathrm{\mathbb{D}}}
\def\P{\mathrm{P}}

\newcounter{znum}
\newcommand{\z}[1]{\addtocounter{znum}{1} \textbf{Задача \arabic{znum}#1. }}

\newcounter{defnum}
\newcommand{\df}[1]{\addtocounter{defnum}{1} \textbf{Определение \arabic{defnum}.} {\it #1}}

\renewcommand{\!}{\textcolor{red}{!}}

\usepackage{wrapfig}
\usepackage{graphicx} % Allows including images
\usepackage{float}


\usepackage[
	% disable,
	colorinlistoftodos, prependcaption, textsize=normalsize,
	linecolor=Dandelion, backgroundcolor=Yellow!25, bordercolor=Dandelion,
	textwidth=100
]{todonotes}

\newcommandx{\note}[2][1=]{\todo[size=footnotesize, #1]{#2}}
\newcommandx{\longnote}[2][1=]{\todo[inline, nolist, linecolor=Gray, backgroundcolor=Gray!25, bordercolor=Gray, #1]{#2}}
\newcommandx{\problem}[2][1=]{\todo[backgroundcolor=red!25, linecolor=red, bordercolor=red, #1]{#2}}
\newcommandx{\question}[2][1=]{\todo[
	noprepend, inline,
	linecolor=RoyalBlue, backgroundcolor=RoyalBlue!25, bordercolor=RoyalBlue, #1]{#2}
}
\newcommandx{\thought}[2][1=]{\todo[
	noprepend, inline,
	linecolor=Orchid, backgroundcolor=Orchid!25, bordercolor=Orchid, #1]{#2}
}

\newcommandx{\checkref}[2][1={check reference validity}, 2=]{
	\todo[linecolor=red, backgroundcolor=red!25, bordercolor=red, size=footnotesize, #2]{#1}
}
\newcommandx{\needref}[2][1={need reference}, 2=]{
	\todo[linecolor=red, backgroundcolor=red!25, bordercolor=red, size=footnotesize, #2]{#1}
}
\newcommand{\EMPH}[1]{{\large \textit{\textbf{#1}}}}


\begin{document}

\pagestyle{empty}

\begin{center}
	\large
	\textbf{Знакомство с теорией вероятностей.}\\
	\normalsize
	О курсе.
\end{center}


Это будет вводный курс про теории вероятностей. 30\% теории, 70\% задач. Многие задачки интересные и вирусные. Можно сказать математические мемасы. Можно потом ходить всех ими озадачивать и люди обычно охотно озадачиваются ;)

Этот курс является пререквизитом для курса по математической статистике, где я расскажу, что значит ''статистически значимый результат'', как доказывать, что ваше лекарство работает, чтобы ученые вам поверили. И как оценить, сколько подопытных крысок вам для этого нужно.

\vspace{1cm}
Приглашаются школьники \EMPH{всех отделений}, \EMPH{9+ класс}.\\
\EMPH{Пререквизит: знакомство с комбинаторикой} (понимание, когда варианты надо перемножать, а когда складывать, и что такое $n!$ и $C_n^k$).\\
\EMPH{Тестовые вопросы:}
\begin{itemize}
	\setlength\itemsep{0em}
	\item Cколько есть способов выбрать из n человек капитана и его заместителя?
	\item А сколько способов выбать двух человек?
	\item Почему ответ разный?
\end{itemize}
\EMPH{Примерная программа:}
\begin{itemize}
	\setlength\itemsep{0em}
	\item независимые события
	\item условные вероятности
	\item формула Байеса
	\item математическое ожидание
\end{itemize}
\EMPH{Критерий зачета:} сдача определенного\note{потом уточню сколько, думаю примерно 10} количества задач.

Тут\problem{сделать qr-код} будут появляться материалы курса по мере того как я буду их генерировать:


\end{document}
